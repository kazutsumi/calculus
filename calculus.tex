\documentclass[11pt, uplatex, dvipdfmx]{jsarticle}
\usepackage{amsmath,amsfonts, bm, braket, setspace, emathEy, enumerate}

\newcommand{\ds}{\displaystyle}
\renewcommand{\dlim}{\lim\limits} %emathEyを使わないなら\newcommand


\resettagform


\pagestyle{plain}



\title{\Huge 微分積分 問題集}


\begin{document}

\maketitle
\thispagestyle{empty}

\newpage


\section{数列の極限}\label{sec:sequence}

\begin{enumerate}[\ref{sec:sequence}.1]
  
  \setlength{\itemsep}{1zh}
  
\item 第 $n$ 項が次の式で表される数列の極限を求めよう.

  \vspace{1zh}
  
  \begin{edaenumerate}<retusuu=3>[(1)]

  \item $\ds \frac{\sin n}{n}$

  \item $\ds \frac{1+\left(-1\right)^n}{n}$

  \item $\ds 1+ \frac{1}{2} + \frac{1}{3}+\cdots + \frac{1}{n} $ 

  \end{edaenumerate}

\item 数列 $\{a_n\}$ が次の漸化式と初期条件を満たすとする.
  \[
    a_{n+1} = \frac{a_{n}}{2}+\frac{1}{a_{n}} ,\quad a_1=-2
  \]
  \begin{enumerate}[(1)]
    \setlength{\itemsep}{1ex}


  \item $\{a_n\}$ が上に有界であることを確認しよう
    
  \item $\{a_n\}$ が狭義単調増加であることを確認しよう.

  \item $\{a_n\}$ の極限を求めよう.

  \item $a_2^2\; , a_3^2\; , a_4^4$ を具体的に計算し,小数表示してみよう.
  \end{enumerate}

\item 次の漸化式と初期条件を満たす数列 $\{a_n\}$ の極限を求めよう.
  \[
    a_{n+1} = \frac{2a_{n}}{3} + \frac{1}{a_{n}^2}, \quad a_1 = 3
  \]
\end{enumerate}



\section{関数の極限}\label{sec:limit}


\begin{enumerate}[\ref{sec:limit}.1]

  \setlength{\itemsep}{1zh}
  
\item 次の極限を求めよう.

  \vspace{1ex}

  \begin{edaenumerate}<retusuu=3>[(1)]

  \item $\ds \lim_{x \to +0} \sqrt{x} \sin \frac{1}{x}$

  \item $\ds \lim_{x \to \infty} \frac{\sin x}{x}$

  \item $\ds \lim_{x \to \infty} \frac{x-\tan^{-1} x}{x}$
    
  \end{edaenumerate}

\item 次の関数 $f$ が $x=0$ で連続となるような定数 $A$ の値を求めよう.

  \vspace{1ex}

  \begin{edaenumerate}[(1)]
    
  \item $\ds f(x)=\left\{
      \begin{array}{cc}
        2x+3& (x > 0)\\
        -3x+A & (x \leqq 0)
      \end{array}
    \right.$

  \item $\ds f(x)=\left\{
      \begin{array}{cc}
        e^{-x} & (x >0)\\
        \cos^{-1}x +A & (x \leqq 0)
      \end{array}
    \right.$

  \item $\ds f(x) = \left\{
      \begin{array}{cc}
        \tan^{-1}\left(x-\frac{1}{2}\right) & (x > 0)\\
        \sin^{-1} A & (x \leqq 0)
      \end{array}
    \right.$
    
  \item $\ds f(x) = \left\{
      \begin{array}{cc}
        \dfrac{\sin 3x}{2x} & (x > 0)\\
        x^2+A & (x \leqq 0)
      \end{array}
    \right.$

  \item $\ds f(x) = \left\{
      \begin{array}{cc}
        e^{-\frac{1}{x}} & (x >0)\\
        A & (x \leqq 0)
      \end{array}
    \right.$
    
  \item $\ds f(x) = \left\{
      \begin{array}{cc}
        e^{-\frac{1}{x^2}}& (x \neq 0)\\
        A & (x=0)
      \end{array}
    \right.$
  \end{edaenumerate}
  
  
\end{enumerate}

\newpage

\section{導関数}\label{sec:derivative}

\begin{enumerate}[\ref{sec:derivative}.1]

  \setlength{\itemsep}{1zh}
  
\item 次の関数 $f$ が $x=0$ で微分可能か否かを判定し,微分可能な
  ら $f'(0)$ の値を求めよう.

  \vspace{1ex}
  
  \begin{edaenumerate}[(1)]
  \item $\ds f(x) = \left\{
      \begin{array}{cc}
        x \log |x| & (x \neq 0)\\
        0 & (x =0)
      \end{array}
    \right.$

  \item $\ds f(x) = \left\{
      \begin{array}{cc}
        \dfrac{1-\cos x}{x} & (x \neq 0)\\
        0 & (x=0)
      \end{array}
    \right.$

  \item $\ds f(x) = \left\{
      \begin{array}{cc}
        x^2 \sin \dfrac{1}{x} & (x \neq 0)\\
        0 & (x=0)
      \end{array}
    \right.$

  \item $\ds f(x) = \left\{
      \begin{array}{cc}
        \sqrt{|x|} \sin \dfrac{1}{x} & (x \neq 0)\\
        0 & (x =0 )
      \end{array}
    \right.$
  \end{edaenumerate}

\item 次の関数 $f$ が開区間 $(-1,1)$ で $C^2$ 級となるような定数 $A, B, C$ の値を求めよう.
  \[
    f(x) = \left\{
      \begin{array}{cc}
        \cos x & (x > 0) \\
        Ax^2+Bx+C & (x \leqq 0)
      \end{array}
    \right.
  \]

\item 次の関数 $f$ に対し,$3$ 次関数 $g(x) =a_3 x^3+a_2x^2 + a_1 x + a_0$ が
  \[
    f(0) = g(0), \quad f'(0) = g'(0), \quad f''(0) = g''(0), \quad f^{(3)}(0) = g^{(3)}(0)
  \]
  を満たすとき,$g(x)$ の係数 $a_0, a_1, a_2, a_3$ の値を求めよう.

  \vspace{1ex}
  
  \begin{edaenumerate}<retusuu=3>[(1)]
  \item $\ds f(x) = \sin x$
        
  \item $\ds f(x) = e^x$

  \item $\ds f(x) = \log (1+x)$
  \end{edaenumerate}
\end{enumerate}



\section{不定形の極限}\label{sec:hospital}

\begin{enumerate}[\ref{sec:hospital}.1]

  \setlength{\itemsep}{1zh}
  
\item 次の関数 $f$ が $x=0$ で連続となるような定数 $A$ の値を求めよう.

  \vspace{1ex}

  \begin{edaenumerate}[(1)]
    
  \item $\ds f(x) = \left\{
      \begin{array}{cc}
        x \log x & (x >0) \\
        A & (x \leqq 0)
      \end{array}
    \right.$
   
  \item $\ds f(x) = \left\{
      \begin{array}{cc}
        x^x & (x >0)\\
        -x + A & (x \leqq 0)
      \end{array}
    \right.$

  \item $f(x) = \left\{
      \begin{array}{cc}
        \dfrac{ x-\sin x}{x^3} & (x \neq 0)\\
        A & (x = 0)
      \end{array}
    \right.$

  \item $\ds f(x) = \left\{
      \begin{array}{cc}
        \dfrac{(1-\cos x)\sin x}{x-\sin x} &  (x \neq 0)\\
        A & (x = 0)
      \end{array}
    \right.$
    
  \end{edaenumerate}

\item 次の関数 $f$ が $x=0$ で微分可能となるような定数 $A, B$ の値を求めよう.

  \vspace{1ex}
  
  \begin{edaenumerate}[(1)]
  \item $\ds f(x) = \left\{
      \begin{array}{cc}
        \dfrac{\sin x}{x} & (x > 0)\\
        Ax +B & (x \leqq 0)
      \end{array}
    \right.$

  \item $\ds f(x) = \left\{
      \begin{array}{cc}
        \dfrac{x-\tan^{-1}x}{x^2} & (x >0)\\
        Ax + B & (x \leqq 0)
      \end{array}
    \right.$
    
  \end{edaenumerate}
  
\end{enumerate}


\newpage

\section{テイラーの定理}\label{sec:taylor}

\begin{enumerate}[\ref{sec:taylor}.1]

  \setlength{\itemsep}{1zh}
  
\item テイラーの定理を用いて次の値の近似値を誤差精度 $0.01$ 以内で求め,
  小数表示しよう.なお,$\pi = 3.141592653589793\cdots$ であることは適
  宜利用しよう.

  \vspace{1ex}

  \begin{edaenumerate}<retusuu=5>[(1)]

  \item $\sin 3$

  \item $\cos 1.6$

  \item $\sqrt{e}$

  \item $\log 1.2$

  \item $\ds \tan^{-1} 0.02$
    
  \end{edaenumerate}

\item $\ds f(x) = \log\frac{1+x}{1-x}$ をうまく利用して,$\log 2$ の小数
  点以下第3位までを確定させよう.

\item $f_n(x) = \sqrt[n]{1+x}$ をうまく利用して,以下の値の小数点
  以下第3位までを確定させよう.

  \vspace{1ex}

  \begin{edaenumerate}<retusuu=3>[(1)]

  \item $\sqrt{1.2}$

  \item $\sqrt[3]{1.01}$

  \item $\sqrt[5]{0.8}$
    
  \end{edaenumerate}

\item 以下の条件を満たす関数 $f$ に対し,$f(0.5)$ の値の小数点以下第 $2$
  位までを確定させよう.
  \[
    f'(x) = e^{-x^2}, \quad f(0)=0
  \]

\item 以下の条件を満たす $C^{\infty}$ 級関数 $f$ に対し,$f$ の $5$ 次マクローリン多項式
  \[
    P(x) = f(0) + f'(0)x + \frac{f''(0)}{2!}x^2 +
    \frac{f^{(3)}(0)}{3!}x^3 + \frac{f^{(4)}(0)}{4!}x^4 + \frac{f^{(5)}(0)}{5!}x^5
  \]
  を求めよう.また,指定された $a$ に対して $P(a)$ の値を求めよう.
  
  \vspace{1ex}

  \begin{enumerate}[(1)]

    \setlength{\itemsep}{1ex}
    
  \item $f'(x) = f(x), \quad f(0) =1$ \quad $\ds (a=1/2)$

  \item $\ds 2f''(x) +5f'(x) -3 f(x)=0, \quad f(0)=1, \; f'(0)= 1/2$ \quad $(a=1)$

  \item $\ds f(x) + \log\left( f'(x)\right) =0, \; f(0)=0$ \quad $(a=0.2)$
    
  \end{enumerate}

\item $C^{\infty}$ 級関数 $f$ が以下の条件を満たすとする.
  \[
    f''(x) = -f(x), \quad f(\pi)=0, \; f'(\pi) = -1
  \]
  このとき,$f$ の $x=\pi$ の周りの $4$ 次テイラー多項式
  \[
    P(x) = f(\pi) + f'(\pi) (x-\pi) + \frac{f''(\pi)}{2!}(x-\pi)^2 + \frac{f^{(3)}(\pi)}{3!}(x-\pi)^3
    + \frac{f^{(4)}(\pi)}{4!}(x-\pi)^4
  \]
  と $P(3)$ の値を求めよう.
    
\end{enumerate}

\newpage

\section{偏微分}\label{sec:partial}

\begin{enumerate}[\ref{sec:partial}.1]

  \setlength{\itemsep}{1zh}
  
\item $f(x,y) = -2x^3-y^2+x^2y+1$ とし,$\ds P=(4,3)$ とする.

  \vspace{1ex}

  \begin{enumerate}[(1)]
    \setlength{\itemsep}{1ex}
    
  \item $P$ における $f$ の勾配 $\nabla f(P)$ を求めよう.

  \item $P$ での $\bm{n} = (n_1, n_2)$ 方向の方向微分
    $\ds \frac{\partial f}{\partial \bm{n}}(P) = \lim_{h \to 0}
    \frac{f(P+h\bm{n}) - f(P)}{h}$ を求めよう.

  \item 平面の単位ベクトル $\bm{n}$ の中で,方向微分
    $\ds \frac{\partial f}{\partial \bm{n}}(P)$ が最小となるものを求めよう.
  \end{enumerate}

\item $f(x,y) = 2x^2+3y^2-1$ とし,$P_0=(3,4)$ とする.

  \vspace{1ex}
  
  \begin{enumerate}[(1)]
    \setlength{\itemsep}{1ex}
    
  \item 平面ベクトル $\bm{d}_0=-\nabla f(P_0)$ を求めよう.

  \item $f(P_0 + \alpha \bm{d}_0)$ を最小にす
    る $\alpha$ を $\alpha_0$ とする.$\alpha_0$ を求めよう.

  \item 内積$\nabla f\left(P_0 + \alpha \bm{d}_0 \right) \cdot
    \bm{d}_0$ が $0$ となる $\alpha$ を求めよう.

  \item $P_1 = P_0  + \alpha_0 \bm{d}_0$ とする.$f(P_1)$ を求めよう.

  \item 上記を繰り返す.つまり,$f(P_k + \alpha \bm{d}_k)$ を最小
    にする $\alpha$ を $\alpha_k$
    とし,
    \[
      P_{k+1} = P_{k} + \alpha_k \bm{d}_k, \quad \bm{d}_k = -\nabla f(P_k) \quad (k=0,1,2,\ldots)
    \]
    とする.$f(P_1), f(P_2), f(P_3), \ldots $ が $f$ の最小値 $-1$ に近
    づいていることを確認しよう.このようにして関数の最小値(極小値)を
    探す方法は最急降下法と呼ばれる.
  \end{enumerate}

\item 平面曲線
  $ 10 x^{3}-3 x^{2} y -26 x \,y^{2}-14 y^{3}-51 x^{2}-134 x y -78
  y^{2}-110 x -117 y -53=0$ 上の点 $(0,-1)$ における接線の方程式を求め
  よう.
  
\end{enumerate}

\section{2変数関数の極値}\label{sec:extremal}

\begin{enumerate}[\ref{sec:extremal}.1]

  \setlength{\itemsep}{1zh}
  
\item 次の関数 $f$ の極値とそれを実現する点を全て求めよう.

  \begin{edaenumerate}[(1)]
  \item $\ds f(x,y) = (x-y)^2+y^3$

  \item $f(x,y) = -x^4-x^2-2xy-y^2$
  \end{edaenumerate}
  
\item 有界閉領域$D=\Set{(x,y) | x^2-y^2 \leqq 1, -1 \leqq y \leqq 1 }$
  における関数 $f(x,y) = x^2-y^3$ の最大値と最小値及びそれらを実現する
  点を全て求めよう.

\item 関数 $\ds f(x,y) = \left\{
    \begin{array}{cc}
      \dfrac{x^4 y^2}{x^8+y^4} & (x,y) \neq (0,0)\\
       0 & (x,y)=(0,0)
    \end{array}
    \right.$ の最大値と最小値を求めよう.
  
\end{enumerate}

\newpage


\section*{解答}

\begin{enumerate}[\ref{sec:sequence}.1]
  \setlength{\itemsep}{1ex}
  
\item
  \begin{enumerate}[(1)]
    \setlength{\itemsep}{1ex}
    
  \item
    $0 \leqq \left| \frac{\sin n}{n}\right| \leqq \frac{1}{n} \to
    0$ より,はさみうちの原理から $\dlim_{n \to \infty} \frac{\sin
      n}{n} = 0.$

  \item
    $0 \leqq \left| \frac{1+(-1)^n}{n}\right| \leqq \frac{2}{n}
    \to 0$ より,はさみうちの原理から
    $\dlim_{n \to \infty}\frac{1+(-1)^n}{n} =0.$

  \item $k$ を $2^k < n \leqq 2^{k+1}$ となる自然数とする.$n \to
    \infty$ のとき $k \to \infty$
    なので,$1+\frac{1}{2}+\frac{1}{3}+ \cdots + \frac{1}{n} \geqq 1 +
    \frac{1}{2} + \left( \frac{1}{3} + \frac{1}{4}\right) + \cdots +
    \left( \cdots + \frac{1}{2^k}\right) \geqq 1 + \frac{1}{2} +
    \left(\frac{1}{4} + \frac{1}{4}\right) + \cdots + \left(
      \frac{1}{2^k} + \cdots + \frac{1}{2^k}\right) = 1+ \frac{k}{2}
    \to \infty \; (k \to
    \infty)$ より,${\ds \lim_{n \to \infty}} \left(1+\frac{1}{2} +
      \frac{1}{3} + \cdots + \frac{1}{n}\right) = \infty.$
  \end{enumerate}

\item
  \begin{enumerate}[(1)]
    \setlength{\itemsep}{1ex}

  \item $a_1=-2<0$ と漸化式から任意の $n$ で $a_n <0$ なので,$\{a_n\}$ は上に有界で
    ある.
    
  \item まず,$a_2=-\frac{3}{2} >a_1$ である.$n \geqq 2$ に対して
    は $a_{n+1}-a_{n} = \frac{2-a_{n}^2}{2a_n} = -
    \frac{(a_{n-1}^2-2)^2}{4a_{n-1}^2 \cdot 2a_n} \geqq 0$ である.ここ
    で,$a_1$ が有理数なのと漸化式の形から $\{a_n\}$ の各項は有理数であ
    る.従って,$a_{n-1}^2 \neq 2$ なので $a_{n+1}-a_n >0$ である.よっ
    て,$\{a_n\}$ は狭義単調増加である.
    
  \item (1), (2) から $\{a_n\}$ は収束するのでその極限値を $\alpha$ と
    する.$\alpha = \dlim_{n \to \infty} a_{n+1} = \dlim_{n \to
      \infty} \left( \frac{a_n}{2}+\frac{1}{a_n}\right) =
    \frac{\alpha}{2}+\frac{1}{\alpha}$ より,$\alpha^2 = 2$ であ
    る.$a_n <0$ より $\dlim_{n \to \infty} a_n = -\sqrt{2}$.

  \item $a_2^2 = \left(-\frac{3}{2}\right)^2 =
    2.25, \; a_3^2 = \left(-\frac{17}{12}\right)^2 =2.0069\cdots, \;
    a_4^2=\left(-\frac{577}{408}\right)^2 =2.000006\cdots$ 
  \end{enumerate}
  
\item $a_1=3>0$ と漸化式から任意の $n$ で $a_n>0$ なので,$\{a_n\}$ は
  下に有界である.さらに,$n \geqq 2$
  に対して$a_n - a_{n+1} = \frac{a_n^3-3}{3a_n^2} =
  \frac{(8a_{n-1}^3+3)(a_{n-1}^3-3)^2}{27a_{n-1}^6 \cdot 3a_{n}^2}
  \geqq 0$ なので,$\{a_n\}$ は($n\geqq 2$ で)単調減少である.よっ
  て,$\{a_n\}$ は収束するのでその極限値を $\alpha$ とすれば,前問と
  同様に漸化式から $\alpha = \frac{2}{3\alpha}+\frac{1}{a^2}$ が成り立
  つので,$\alpha^3=3$である.従って,$\dlim_{n \to \infty} a_n
  = \sqrt[3]{3}$.
\end{enumerate}

\begin{enumerate}[\ref{sec:limit}.1]

  \setlength{\itemsep}{1ex}
  
\item
  \begin{enumerate}[(1)]
    \setlength{\itemsep}{1ex}
    
  \item
    $0 \leqq \left| \sqrt{x}\sin \frac{1}{x}\right| \leqq \sqrt{x} \to
    0 \; (x \to +0)$ より,$\dlim_{x \to +0} \sqrt{x} \sin \frac{1}{x} =0$.
    
  \item
    $0 \leqq \left| \frac{\sin x}{x} \right| \leqq \frac{1}{x} \to 0
    \; (x \to \infty)$ より,$\dlim_{x \to \infty} \frac{\sin x}{x}=0$.

  \item $\frac{x-\tan^{-1}{x}}{x} = 1 -\frac{\tan^{-1}x}{x}$
    と $\left|\frac{\tan^{-1}x}{x}\right| <
    \frac{\pi/2}{x} \to 0 \; ( x \to \infty)$
    より,$\dlim_{x \to \infty}\frac{x-\tan^{-1}x}{x}=1$.
  \end{enumerate}

\item 
  \begin{enumerate}[(1)]
    \setlength{\itemsep}{1ex}
    
  \item
    $\dlim_{x \to +0} f(x)= 3, \dlim_{x \to -0} f(x) = A =
    f(0)$ より,$A=3$.

  \item
    $\dlim_{x \to +0} f(x)  =1,
    \dlim_{x \to -0} f(x) = \frac{\pi}{2} +A = f(0)$ よ
    り,$A=1-\frac{\pi}{2}$.

  \item
    $\dlim_{x \to +0} f(x) = \tan^{-1}\left(-\frac{1}{2}\right),
    \dlim_{x \to -0} f(x) = \sin^{-1} A = f(0)$ よ
    り,$\tan^{-1}\left(-\frac{1}{2}\right) = \sin^{-1}A$ であ
    る.$X=\tan^{-1}\left(-\frac{1}{2}\right)$ とおくと
    $\tan X = -\frac{1}{2}$ かつ $-\frac{\pi}{2} < X < 0$ なの
    で,$A=\sin \left(\tan^{-1}\left(-\frac{1}{2}\right)\right) = \sin
    X = -\frac{1}{\sqrt{5}}$.

  \item
    $\dlim_{x \to +0} f(x) = \dlim_{x \to +0} \frac{\sin
      3x}{2x}=\frac{3}{2}, \; \dlim_{x \to -0} f(x) = A = f(0)$ よ
    り,$A=\frac{3}{2}$.

  \item
    $\dlim_{x \to +0} f(x) = \dlim_{x \to +0} e^{-\frac{1}{x}} = 0,
    \dlim_{x \to -0} f(x) = A = f(0)$ より,$A=0$.

  \item $\dlim_{x \to 0} f(x) = \dlim_{x \to 0} e^{-\frac{1}{x^2}} = 0, f(0)=A$ より,$A=0$.
  \end{enumerate}

\end{enumerate}

\begin{enumerate}[\ref{sec:derivative}.1]
  \setlength{\itemsep}{1ex}
  
\item
  \begin{enumerate}[(1)]
    \setlength{\itemsep}{1ex}
    
  \item
    $\dlim_{x \to 0} \frac{f(x)-f(0)}{x} = \dlim_{x \to 0}\log|x|
    =-\infty$ より,微分不可能.
    
  \item
    $\dlim_{x \to 0} \frac{f(x)-f(0)}{x} = \dlim_{x \to 0}
    \frac{1-\cos x}{x^2}=\frac{1}{2}$ より,微分可能で $f'(0) =
    \frac{1}{2}$.
    
  \item
    $0 \leqq \left| \frac{f(x)-f(0)}{x}\right| = \left| x \sin
      \frac{1}{x}\right| \leqq |x| \to 0 \; (x \to 0)$ より,微分可能で $f'(0)=0$.
    
  \item $x>0$ で 
    $\frac{f(x)-f(0)}{x} = \frac{1}{\sqrt{x}} \sin \frac{1}{x}$ よ
    り,$\dlim_{x \to +0} \frac{f(x)-f(0)}{x}$ が存在しないから微分不可
    能.
  \end{enumerate}

\item $f$ は $x=0$ で連続なので,$C= f(0) = \dlim_{x \to +0} f(x) =
  \cos 0=1$ である.また,$f$ は $x=0$ で微分可能なので
  $f'(0) = \dlim_{x \to +0} \frac{f(x)-f(0)}{x} = \dlim_{x \to -0}
  \frac{f(x)-f(0)}{x}$
  である.$\dlim_{x \to +0} \frac{f(x)-f(0)}{x} = \dlim_{x \to +0}
  \frac{\cos x -1}{x} = 0, \dlim_{x \to -0} \frac{f(x)-f(0)}{x} =
  \dlim_{x \to -0} (Ax+B) = B$ より $f'(0) = B = 0$ である.さら
  に,$f'$ は $x=0$ で微分可能なので
  $f''(0) = \dlim_{x \to +0} \frac{f'(x)-f'(0)}{x} = \dlim_{x \to -0}
  \frac{f'(x)-f'(0)}{x}$
  である.$\dlim_{x \to +0} \frac{f'(x)-f'(0)}{x} = \dlim_{x \to
    +0}\frac{-\sin x}{x}=-1, \dlim_{x \to -0}
  \frac{f'(x)-f'(0)}{x}=\dlim_{x \to -0}2A = 2A$ より
  $f''(0)=-1, A=-\frac{1}{2}$
  である.最後に,$\dlim_{x \to +0} f''(x) = \dlim_{x \to +0} -\cos x
  = -1, \dlim_{x \to -0} f''(x) = \dlim_{x \to -0}
  \left(-\frac{1}{2}x^2+1\right)'' = -1$ より
  $\dlim_{x \to 0} f''(x) = f''(0)$ なので,$f''$ は $x=0$ で連続である.

\item $g(0)=a_0, \, g'(0)=a_1, \, g''(0)=2a_2, \,
  g^{(3)}(0)=6a_3$ より
  $a_0=g(0)=f(0), \, a_1= g'(0) = f'(0), \, a_2 = g''(0)/2 = f''(0)/2,
  \, a_3=g^{(3)}(0)/6=f^{(3)}(0)/6$ である.
  \begin{enumerate}[(1)]
    \setlength{\itemsep}{1ex}
    
  \item
    $a_0 = \sin 0 =0, \, a_1 = \cos 0 = 1, \, a_2 = -\frac{1}{2}\sin 0 =0, \, a_3 =
    -\frac{1}{6}\cos 0 = -\frac{1}{6}$

  \item
    $a_0 = e^0=1, \, a_1=e^0=1, \, a_2= \frac{e^0}{2}=\frac{1}{2}, \,
    a_3=\frac{e^0}{6}=\frac{1}{6}$

  \item
    $a_0=\log (1+0) =0, \, a_1 = (1+0)^{-1} = 1, \, a_2 = \frac{-(1+0)^{-2}}{2}=
    -\frac{1}{2}, \, a_3 = \frac{2(1+0)^{-3}}{6} = \frac{1}{3}$
  \end{enumerate}

\end{enumerate}

\begin{enumerate}[\ref{sec:hospital}.1]
  \setlength{\itemsep}{1ex}
  
\item $(\ast)$ の箇所でロピタルの定理を用いている.
  \begin{enumerate}[(1)]
    \setlength{\itemsep}{1ex}
    
  \item
    $A = f(0) = \dlim_{x \to +0} f(x) = \dlim_{x \to +0} x \log x =
    \dlim_{x \to +0} \frac{\log x}{\frac{1}{x}} \overset{(\ast)}{=} 0$

  \item
    $A = f(0) = \dlim_{x \to +0} f(x) = \dlim_{x \to +0} x^x =
    \dlim_{x \to +0} e^{x \log x} = e^{\dlim_{x \to +0} x \log x} \overset{(\ast)}{=} e^0=1$ 

  \item
    $A= f(0) = \dlim_{x \to 0}f(x) = \dlim_{x \to 0} \frac{x-\sin
      x}{x^3} \overset{(\ast)}{=}\frac{1}{6}$

  \item
    $A=f(0) = \dlim_{x \to 0} f(x) = \dlim_{x \to 0} \frac{(1-\cos
      x)\sin x}{x-\sin x} \overset{(\ast)}{=} 3$
    
  \end{enumerate}

\item $(\ast)$ の箇所でロピタルの定理を用いている.
  \begin{enumerate}[(1)]
    \setlength{\itemsep}{1ex}
    
  \item $f$ は $x=0$ で連続なので
    $B=f(0) = \dlim_{x \to +0} f(x) = \dlim_{x \to +0}\frac{\sin x}{x}
    = 1$ である.$f$ は $x=0$ で微分可能なので
    $\dlim_{x \to +0} \frac{f(x)-f(0)}{x} = \dlim_{x \to +0}
    \frac{\sin x - x}{x^2} \overset{(\ast)}{=}  0$ と $\dlim_{x \to -0}
    \frac{f(x)-f(0)}{x} = \dlim_{x \to -0} A = A$ が一致するから $A=0$.
    
  \item $f$ は $x=0$
    で連続なので$B=f(0) = \dlim_{x \to +0} f(x) = \dlim_{x \to +0}
    \frac{x-\tan^{-1}x}{x^2} \overset{(\ast)}{=} 0$ であ
    る.$f$ は $x=0$
    で微分可能なので$\dlim_{x \to +0}\frac{f(x)-f(0)}{x} = \dlim_{x
      \to +0} \frac{x-\tan^{-1}x}{x^3} \overset{(\ast)}{=} \frac{1}{3}$
    と$\dlim_{x \to -0} \frac{f(x)-f(0)}{x} = \dlim_{x \to -0} A=A$ が
    一致するから $A=\frac{1}{3}$.
    
  \end{enumerate}
\end{enumerate}

\newpage

\begin{enumerate}[\ref{sec:taylor}.1]
  \setlength{\itemsep}{1ex}
  
\item
  \begin{enumerate}[(1)]

    \setlength{\itemsep}{1ex}
    
  \item $f(x) = \sin x$ とすると $\sin 3 = f(3)$ であり,$3$ は $\pi$
    に近い.$\pi$ に近い $x$ に対し $f(x)$ を $3$ 次式
    $g(x) = {\ds \sum_{n=0}^{3}} \frac{f^{(n)}(\pi)}{n!} (x-\pi)^n=
    -(x-\pi) + \frac{(x-\pi)^3}{6} $ で近似し,その誤差を $R(x) = f(x)
    - g(x)$
    とすると,テイラーの定理から$R(3) = \frac{f^{(4)}(c)}{4!}(3-\pi)^4
    = \frac{\sin c}{24}(3-\pi)^4, \; 3 < c < \pi$ を満たす $c$ があ
    る.$|\sin c|
    <1$なので,$|R(3)| < \frac{(\pi-3)^4}{24}<
    \frac{(3.15-3)^4}{24}=0.00625 < 0.01$ より近似値 $g(3)=
    0.141\cdots$ の誤差は $0.01$ 以内.

  \item $f(x) = \cos x$ とすると $\cos 1.6 = f(1.6)$ であ
    り,$1.6$ は $\frac{\pi}{2}=1.57\cdots $ に近い.$f(x)$ と $3$ 次近
    似式
    $g(x) = {\ds \sum_{n=0}^{3}}
    \frac{f^{(n)}\left(\frac{\pi}{2}\right)}{n!}\left(x-\frac{\pi}{2}\right)^n
    = -\left(x-\frac{\pi}{2}\right) +
    \frac{\left(x-\frac{\pi}{2}\right)^3}{6}$ との誤差を
    $R(x) = f(x) - g(x)$
    とすると,テイラーの定理から$R(1.6) =
    \frac{f^{(4)}(c)}{4!}\left(1.6-\frac{\pi}{2}\right)^4 = \frac{
      \cos c}{24}\left( 1.6-\frac{\pi}{2}\right)^4 , \, \frac{\pi}{2}
    < c < 1.6$ を満たす $c$ がある.$|\cos c|
    <1$なので,$|R(1.6)| < \frac{\left(1.6 -
        \frac{\pi}{2}\right)^4}{24} < 0.01$ より近似
    値 $g(1.6)=-0.029\cdots$ の誤差は $0.01$ 以内.

  \item $f(x)=e^x$ とすると $\sqrt{e} = f\left(\frac{1}{2}\right)$ であ
    る.$f(x)$ と $3$ 次近似式
    $g(x) = {\ds \sum_{n=0}^{3}}\frac{f^{(n)}(0)}{n!}x^n = 1 + x +
    \frac{x^2}{2} + \frac{x^3}{6} $ との誤差を $R(x) = f(x)
    - g(x)$
    とすると,テイラーの定理から$R\left(\frac{1}{2}\right) =
    \frac{f^{(4)}(c)}{4!}\left( \frac{1}{2}\right)^4= \frac{e^c}{384},
    \, 0 < c < \frac{1}{2}$ を満たす $c$
    がある.$e^c < e^{\frac{1}{2}} < 3$ なの
    で,$\left|R\left(\frac{1}{2}\right)\right| < \frac{3}{384} <
    0.01$ より近似値
    $g\left(\frac{1}{2}\right) = \frac{79}{48}= 1.6458\cdots$ の誤差
    は $0.01$ 以内.

  \item $f(x) = \log(1+x)$ とすると $\log 1.2 = f(0.2)$ である.$f(x)$
    と $2$ 次近似式
    $g(x) = {\ds \sum_{n=0}^{2}}\frac{f^{(n)}(0)}{n!}x^n= x
    -\frac{x^2}{2}$ との誤差を $R(x)=f(x)-g(x)$ とすると,テイ
    ラーの定理から$R(0.2) = \frac{f^{(3)}(c)}{3!}(0.2)^3 =
    \frac{1}{375(1+c)^3}, \, 0 < c < 0.2$ を満たす $c$ があ
    る.$\frac{1}{(1+c)^3} <1$ なので,$|R(0.2)| < \frac{1}{375}
    <0.01$ より近似値 $g(0.2)= 0.18$ の誤差は $0.01$ 以内.

  \item $f(x) = \tan^{-1} x$ とすると $\tan^{-1} 0.02 = f(0.02)$ であ
    る.$1$ 次近似式 $g(x)= f(0) + f'(0)x= x$ と $f(x)$ の誤差
    を $R(x)=f(x)-g(x)$とすると,テイラーの定理から
    $R(0.02) = \frac{f^{(2)}(c)}{2} (0.02)^2=
    -\frac{c}{1250(c^2+1)^2}, \, 0 < c < 0.02$ を満たす $c$ があ
    る.$\frac{c}{(c^2+1)^2} < \frac{0.02}{(0.02^2+1)^2} < 0.02$なの
    で,$|R(0.02)| < \frac{0.02}{1250} < 0.01$ より近似
    値 $g(0.02)=0.02$ の誤差は $0.01$ 以内.
    
  \end{enumerate}

\item $\log 2 = f\left(\frac{1}{3}\right)$ である.$f$ とその $7$ 次近
  似式 $g(x) = 2x+\frac{2}{3}x^3+\frac{2}{5}x^5+\frac{2}{7}x^7$ との誤
  差を $R(x) = f(x) - g(x)$ とすると,テイラーの定理か
  ら
  $R\left(\frac{1}{3}\right) = \frac{1}{52488}\left(\frac{1}{(1-c)^8}
    - \frac{1}{(1+c)^8}\right), 0 < c < \frac{1}{3}$ を満たす $c$ があ
  る.$0 < R\left(\frac{1}{3}\right) < \frac{255}{524288}$ なの
  で,$g\left(\frac{1}{3}\right) < f\left(\frac{1}{3}\right) <
  g\left(\frac{1}{3}\right) + \frac{255}{524288}$ である.よっ
  て,$0.6931 < \log 2 < 0.6937$ より小数第3位までが $0.693$ と確定す
  る.

\item
  \begin{enumerate}[(1)]
    
    \setlength{\itemsep}{1ex}
    
  \item $\sqrt{1.2}=f_2(0.2)$ である.$f_2$ と $2$ 次近似式
    $g(x) = 1+\frac{x}{2}-\frac{x^2}{8}$ との誤差を $R(x) =
    f_2(x)-g(x)$ とすると,テイラーの定理から
    $R(0.2) = \frac{1}{2000(1+c)^{5/2}}, 0 < c < 0.2$ を満たす $c$ があ
    る.$0 < R(0.2) < \frac{1}{2000}$
    なので,$g(0.2) < f(0.2) < g(0.2)+ \frac{1}{2000}$ である.よっ
    て,$1.095 < \sqrt{1.2} < 1.0955$ より小数点第3位までが $1.095$ と
    確定する.
    
  \item $\sqrt[3]{1.01}=f_3(0.01)$ である.$f_3$ とその $1$ 次近似
    式 $g(x) = 1+\frac{x}{3}$ との誤差を $R(x) = f_3(x)-g(x)$ とすると,
    テイラーの定理から $R(0.01) = -\frac{1}{90000(1+c)^{5/3}}, 0 < c <
    0.01$ を満たす $c$ がある.$-\frac{1}{90000} < R(0.01) < 0$ なの
    で,$g(0.01) -\frac{1}{90000} < f_3(0.01) < g(0.01)$
    より,$1.00332 < \sqrt[3]{1.01} < 1.00334$ から小数点第3位まで
    が $1.003$ と確定する.
    
  \item $\sqrt[5]{0.8} = f_5(-0.2)$ である.$3$ 次近似式
    $g(x) = 1+\frac{x}{5} - \frac{2}{25}x^2+\frac{6}{125}x^3$ で近似す
    れば,テイラーの定理から誤差は $-\frac{6}{15625} < R(-0.2) <0$ と評
    価できる.よって,$ 0.9564 < \sqrt[5]{0.8} < 0.9568$ より,小数
    第3位までが $0.956$ と確定する.
    
  \end{enumerate}

\item $x=0$ を中心とする $f$ の $6$ 次近似式を $g(x)$ とし,その誤差
  を $R(x)=f(x)-g(x)$ とする.$f'(x)=e^{-x^2}$ より $f$ の高階導関数
  は $f''(x) = -2xe^{-x^2}, f^{(3)}(x)=2e^{-x^2}(2x^2-1), \ldots$ と逐
  次計算できるので,$f(0)=0$ と合わせて $f$ の $6$ 次近似式は
  $g(x)= x - \frac{x^3}{3} + \frac{x^5}{10}$ と求められる.テイラーの定
  理より
  $R(0.5) = \frac{f^{(7)}(c)}{7!}(0.5)^7=
  \frac{e^{-c^2}(8c^6-60c^4+90c^2-15)}{80640}, 0 < c < 0.5$ を満た
  す $c$ がある.ここで $3$ 次関数 $p(t) = 8t^3-60t^2+90t-15$ を考える
  と,$0<c^2<0.25$ なので $-15 < p(c^2) < 3.875$
  より,$|R(0.5)| < \left|\frac{e^{-c^2}p(c^2)}{80640}\right| < \left|
    \frac{p(c^2)}{80640}\right| < \frac{15}{80640} = \frac{1}{5376}$ で
  ある.よって,$g(0.5) - \frac{1}{5376} < f(0.5) <
  g(0.5)+\frac{1}{5376}$ より $0.4612 < f(0.5) < 0.4617$ である.これよ
  り小数第2位までが $0.46$ と確定する.

\item
  \begin{enumerate}[(1)]
    
    \setlength{\itemsep}{1ex}
    
  \item $f(x) = f'(x) = f''(x) = f^{(3)}(x) = f^{(4)}(x) = f^{(5)}(x)$
    より $f(0) = f'(0) = f''(0)=f^{(3)}(0)=f^{(4)}(0)=f^{(5)}(0)=1$ で
    ある.これより
    $P(x) = 1 + x + \frac{x^2}{2} + \frac{x^3}{6} + \frac{x^4}{24} +
    \frac{x^5}{120}, \, P\left(\frac{1}{2}\right) =
    \frac{6331}{3840}=1.6486\cdots$.
    
  \item まず,$f''(0) = \frac{-5f'(0) +3f(0)}{2}=\frac{1}{4}$ である.
    次に,$2f^{(3)}(x) + 5f''(x) - 3f'(x)=0$ より
    $f^{(3)}(0) = \frac{-5f''(0)+3f'(0)}{2}=\frac{1}{8}$ である.以下同
    様に
    $f^{(4)}(0) = \frac{-5f^{(3)}(0)+3f''(0)}{2} = \frac{1}{16},
    f^{(5)}(0) = \frac{-2f^{(4)}(0)+3f^{(3)}(0)}{2}=\frac{1}{32}$ であ
    る.これより
    $P(x) = 1+ \frac{x}{2} + \frac{x^2}{8} + \frac{x^3}{48} +
    \frac{x^4}{384} + \frac{x^5}{3840}, P(1) = \frac{6331}{3840} = 1.6486\cdots$.
    
  \item $f'(x) = e^{-f(x)}$ より $f'(0) = e^{-f(0)} = 1$,
    $f'(x) + \frac{f''(x)}{f'(x)}=0$ から $f''(x) =
    -f'(x)^2$ より $f''(0) = -f'(0)^2=-1$ ,
    $f^{(3)}(x) = -2f'(x)f''(x)= 2f'(x)^3$ より $f^{(3)}(0)
    =2f'(0)^3=2$ , $f^{(4)}(x) = 6f'(x)^2 f''(x) =
    -6f''(x)^2$ より $f^{(4)}(0) = -6f''(0)^2=-6$ ,
    $f^{(5)}(x) = -12f''(x)f^{(3)}(x)$ より
    $f^{(5)}(0) = -12 f''(0) f^{(3)}(0) =24$. これより
    $P(x) = x-\frac{x^2}{2} +
    \frac{x^3}{3}-\frac{x^4}{4}+\frac{x^5}{5}, \;
    P(0.2)=0.1823\cdots$.
    
  \end{enumerate}

\item
  $f''(\pi) = -f(\pi)=0, \, f^{(3)}(\pi)=-f'(\pi)=1, \, f^{(4)}(\pi)=
  -f''(\pi)=0$ より $P(x) = -(x-\pi) + \frac{(x-\pi)^3}{6}, P(3) = 0.141\cdots$.
  
\end{enumerate}


\begin{enumerate}[\ref{sec:partial}.1]
  \setlength{\itemsep}{1ex}
  
\item
  \begin{enumerate}[(1)]
    \setlength{\itemsep}{1ex}
    
  \item $\nabla f(x,y) = \left( -6x^2+2xy, x^2-2y\right)$ より
    $\nabla f(P) = \nabla f(4,3))=(-72,10)$.

  \item
    $\frac{\partial f}{\partial \bm{n}}(P)= \dlim_{h \to
      0}\frac{-2(n_1 h + 4)^3 - (n_2 h+3)^2 +
      (n_1h+4)^2(n_2h+3)+89}{h} = -72 n_1 + 10 n_2$.

    
  \item (2) より
    $\frac{\partial f}{\partial \bm{n}}(P) = \nabla f(P) \cdot \bm{n}$
    なので,$\bm{n} = -\frac{\nabla f(P)}{ |\nabla f(P)|} =
    \frac{1}{\sqrt{1321}}(36,-5)$ のとき最小.
  \end{enumerate}

\newpage
  
 \item
   \begin{enumerate}[(1)]
     \setlength{\itemsep}{1ex}
     
   \item $\nabla f(x,y) = (4x, 6y)$ より $\bm{d}_0 = -\nabla f(3,4) = (-12, -24)$.

   \item
     $f(P_0 + \alpha \bm{d}_0 )=f(3-12\alpha, 4-24\alpha) =
     2016\alpha^2-720\alpha+65$
     なので,$\frac{d}{d\alpha} f(P_0 + \alpha \bm{d}_0) =
     4032\alpha-720=0$ を解いて $\alpha_0=\frac{5}{28}$.

   \item
     $\nabla f(3-12\alpha, 4-24\alpha) \cdot (-12,-24) =
     4032\alpha-720 =0$ を解いて $\alpha = \alpha_0=\frac{5}{28}$.

   \item $f(P_1) = f\left(\frac{6}{7}, -\frac{2}{7} \right) = \frac{5}{7}$.

   \item
     $ f(P_1) = \frac{5}{7}=0.7142\cdots , \;  f(P_2) = f\left( \frac{6}{77}, \frac{8}{77}\right) =
     -\frac{515}{539}=-0.955\cdots, \; f(P_3) = f\left(\frac{12}{539},
       -\frac{4}{539}\right) = -\frac{41455}{41503} = -0.998\cdots$ と
     確かに $-1$ に近づいている.
   \end{enumerate}

 \item $f(x,y) = 10x^3-3x^2y-\cdots$
   とおく.$f(0,-1) = 0, \, f_x(0,-1) = -2 \neq 0$ なので陰関数定理か
   ら $x=0$ の近くで $f(x, g(x))=0,\, g(0)=-1$ を満たす微分可能な関
   数 $g$ があり,$g'(0) = -\frac{f_x(0,-1)}{f_y(0,-1)}$ である.つま
   り,$(0,-1)$ の近くで曲線 $f(x,y)=0$ は $y=g(x)$ と一致するの
   で,$(0,-1)$ における接線の方程式は $y-g(0)=g'(0)(x-0)$ である.これ
   を整理して $f_x(0,-1)(x-0) + f_y(0,-1)(y+1) =0$ より接線の方程式
   は $-2x-3y-3=0$.

 \end{enumerate}

 \begin{enumerate}[\ref{sec:extremal}.1]
   \setlength{\itemsep}{1ex}
   
 \item
   \begin{enumerate}[(1)]
     \setlength{\itemsep}{1ex}
     
   \item $f$ の停留点は $(0,0)$ のみで,$(0,0)$ におけるヘシア
     ン $H(x,y) = f_{xx} f_{yy} - f_{xy}^2=12y$ の値は $0$ なので,ヘシアン
     による極値判定法は役に立たない.$f(t,t) = t^3$ より,$t>0$ に対し
     て $f(-t,-t)<0 = f(0,0) < f(t,t)$ なので,$(0,0)$ のどんな近傍で
     も $f(0,0)$ は最大にも最小にもならない.つまり,$f(0,0)$ は極値で
     はないので,$f$ は極値を持たない.

   \item $f$ の停留点は $(0,0)$ のみで,$(0,0)$ におけるヘシア
     ン $H(x,y) = f_{xx} f_{yy} - f_{xy}^2=24x^2$ の値は $0$ なので,ヘ
     シアンによる極値判定法は役に立たない.一方で,$f(x,y) = -x^4 -
     (x+y)^2 \leqq 0$ なので明らかに $f$ は $(0,0)$ で極大値 $0$ をとる.
   \end{enumerate}

 \item $D$ の内部での停留点と境界上で極値をとる点における値を比べ
   る.$f$ の停留点は $(0,0)$ のみでこれは $D$ の内部にあ
   り,$f(0,0)=0$ である.$D$ は双曲線 $C :x^2-y^2=1$ と$2$ 直線 $L_1
   : y=1$ と $L_2: y=-1$ で囲まれる閉領域である.ラグランジュの未定乗数
   法から,$f$ は $C$ 上では $4$ 点
   $(x, y) =(\pm 1, 0), \left(\pm \frac{\sqrt{13}}{3},
     \frac{2}{3}\right) \in D$
   で極値をとり,$f(\pm 1, 0) =1, \; f\left(\pm \frac{\sqrt{13}}{3},
     \frac{2}{3}\right) = \frac{31}{27}$
   である.また,$f(x, \pm 1) = x^2 \mp 1$ なので,$2$ 直線 $L_1, L_2$
   上ではそれぞれ $(0,1), (0,-1)$ で極値をとり,$f(0,\pm 1) = \mp 1$ で
   ある.さらに,$C, L_1$ の交点 $(\pm \sqrt{2}, 1)$ での値は
   $f(\pm \sqrt{2}, 1) = 1$ で, $C, L_2$ の交点 $(\pm \sqrt{2}, -1)$
   での値は $f(\pm \sqrt{2}, -1) =3$ である.以上か
   ら,$f$ は $(0,1)$ で最小値 $-1$ をとり,$(\pm \sqrt{2}, -1)$ で最大
   値 $3$ をとる.

 \item 明らかに $f(x,y) \geqq 0$ かつ $f(x,0)=f(y,0)=0$ なので $f$ の最
   小値は $0$ である.また,$(x,y) \neq (0,0)$ での $f$ の停留点は $t
   \neq 0$ によって$(0,t), (t,0), (t,\pm t^2)$ と書ける点全てなの
   で,$f(t, \pm t^2) = \frac{1}{2} \; (t \neq 0) $ が $f$ の最大値の候
   補である.ここで,$k
   >\frac{1}{2}$のとき,$k - f(x,y) = \frac{kx^8-x^4y^2+ky^4}{x^8+y^4}
   = \frac{k(x^4-y^2)^2+(2k-1)x^4y^2}{x^8+y^4} >0$ なので $f(x,y) = k$
   を満たす $(x,y)$ は存在しない.よって,$f(x,y) \leqq \frac{1}{2}$ な
   ので $\frac{1}{2}$ は $f$ の最大値である.
 \end{enumerate}

\end{document}
