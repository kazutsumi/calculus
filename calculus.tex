\documentclass[11pt, uplatex, dvipdfmx, twoside]{jsarticle}
\usepackage{graphicx}
\usepackage{amsmath,amsfonts, bm, braket, setspace, emathEy, enumitem}

\newcommand{\ds}{\displaystyle}


\pagestyle{plain}



\title{\Huge 微分積分 問題集}

\begin{document}


\maketitle
\thispagestyle{empty}

\newpage



\section{数列の極限}

\begin{enumerate}[label=\arabic{section}.\arabic*]
  
  \setlength{\itemsep}{1zh}
  
\item 第 $n$ 項が次の式で表される数列の極限を求めよう.

  \vspace{1zh}
  
  \begin{edaenumerate}<retusuu=3>[(1)]

  \item $\ds \frac{\sin n}{n}$

  \item $\ds \frac{1+\left(-1\right)^n}{n}$

  \item $\ds 1+ \frac{1}{2} + \frac{1}{3}+\cdots + \frac{1}{n} $ 

  \end{edaenumerate}

\item 数列 $\{a_n\}$ が次の漸化式と初期条件を満たすとする.
  \[
    a_{n+1} = \frac{a_{n}}{2}+\frac{1}{a_{n}} ,\quad a_1=-2
  \]
  \begin{enumerate}[label=(\arabic*)]
    \setlength{\itemsep}{1ex}
  \item $\{a_n\}$ が狭義単調増加であることを確認しよう.

  \item $\{a_n\}$ が上に有界であることを確認しよう.

  \item $\{a_n\}$ の極限を求めよう.

  \item $a_2^2\; , a_3^2\; , a_4^4$ を具体的に計算し,小数表示してみよう.
  \end{enumerate}

\item 次の漸化式と初期条件を満たす数列 $\{a_n\}$ の極限を求めよう.
  \[
    a_{n+1} = \frac{2a_{n}}{3} + \frac{1}{a_{n}^2}, \quad a_1 = 3
  \]
\end{enumerate}



\section{関数の極限}


\begin{enumerate}[label=\arabic{section}.\arabic*]

  \setlength{\itemsep}{1zh}
  
\item 次の極限を求めよう.

  \vspace{1ex}

  \begin{edaenumerate}<retusuu=3>[(1)]

  \item $\ds \lim_{x \to +0} \sqrt{x} \sin \frac{1}{x}$

  \item $\ds \lim_{x \to \infty} \frac{\sin x}{x}$

  \item $\ds \lim_{x \to \infty} \frac{x-\tan^{-1} x}{x}$
    
  \end{edaenumerate}

\item 次の関数 $f$ が $x=0$ で連続となるような定数 $A$ の値を求めよう.

  \vspace{1ex}

  \begin{edaenumerate}[(1)]
    
  \item $\ds f(x)=\left\{
      \begin{array}{cc}
        2x+3& (x > 0)\\
        -3x+A & (x \leqq 0)
      \end{array}
    \right.$

  \item $\ds f(x)=\left\{
      \begin{array}{cc}
        e^{-x} & (x >0)\\
        \cos^{-1}x +A & (x \leqq 0)
      \end{array}
    \right.$

  \item $\ds f(x) = \left\{
      \begin{array}{cc}
        \tan^{-1}\left(x-\frac{1}{2}\right) & (x > 0)\\
        \sin^{-1} A & (x \leqq 0)
      \end{array}
    \right.$
    
  \item $\ds f(x) = \left\{
      \begin{array}{cc}
        \dfrac{\sin 3x}{2x} & (x > 0)\\
        x^2+A & (x \leqq 0)
      \end{array}
    \right.$

  \item $\ds f(x) = \left\{
      \begin{array}{cc}
        e^{-\frac{1}{x}} & (x >0)\\
        A & (x \leqq 0)
      \end{array}
    \right.$
    
  \item $\ds f(x) = \left\{
      \begin{array}{cc}
        e^{-\frac{1}{x^2}}& (x \neq 0)\\
        A & (x=0)
      \end{array}
    \right.$
  \end{edaenumerate}
  
  
\end{enumerate}

\newpage

\section{導関数}

\begin{enumerate}[label=\arabic{section}.\arabic*]

  \setlength{\itemsep}{1zh}
  
\item 次の関数 $f$ が $x=0$ で微分可能か否かを判定し,微分可能な
  ら $f'(0)$ の値を求めよう.

  \vspace{1ex}
  
  \begin{edaenumerate}[(1)]
  \item $\ds f(x) = \left\{
      \begin{array}{cc}
        x \log |x| & (x \neq 0)\\
        0 & (x =0)
      \end{array}
    \right.$

  \item $\ds f(x) = \left\{
      \begin{array}{cc}
        \dfrac{1-\cos x}{x} & (x \neq 0)\\
        0 & (x=0)
      \end{array}
    \right.$

  \item $\ds f(x) = \left\{
      \begin{array}{cc}
        x^2 \sin \dfrac{1}{x} & (x \neq 0)\\
        0 & (x=0)
      \end{array}
    \right.$

  \item $\ds f(x) = \left\{
      \begin{array}{cc}
        \sqrt{|x|} \sin \dfrac{1}{x} & (x \neq 0)\\
        0 & (x =0 )
      \end{array}
    \right.$
  \end{edaenumerate}

\item 次の関数 $f$ が開区間 $(-1,1)$ で $C^2$ 級となるような定数 $A, B, C$ の値を求めよう.
  \[
    f(x) = \left\{
      \begin{array}{cc}
        \cos x & (x > 0) \\
        Ax^2+Bx+C & (x \leqq 0)
      \end{array}
    \right.
  \]

\item 次の関数 $f$ に対し,$3$ 次関数 $g(x) =a_3 x^3+a_2x^2 + a_1 x + a_0$ が
  \[
    f(0) = g(0), \quad f'(0) = g'(0), \quad f''(0) = g''(0), \quad f^{(3)}(0) = g^{(3)}(0)
  \]
  を満たすとき,$g(x)$ の係数 $a_0, a_1, a_2, a_3$ の値を求めよう.

  \vspace{1ex}
  
  \begin{edaenumerate}<retusuu=3>[(1)]
  \item $\ds f(x) = \sin x$
        
  \item $\ds f(x) = e^x$

  \item $\ds f(x) = \log (1+x)$
  \end{edaenumerate}
\end{enumerate}



\section{ 不定形の極限}

\begin{enumerate}[label=\arabic{section}.\arabic*]

  \setlength{\itemsep}{1zh}
  
\item 次の関数 $f$ が $x=0$ で連続となるような定数 $A$ の値を求めよう.

  \vspace{1ex}

  \begin{edaenumerate}[(1)]
    
  \item $\ds f(x) = \left\{
      \begin{array}{cc}
        x \log x & (x >0) \\
        A & (x \leqq 0)
      \end{array}
    \right.$
   
  \item $\ds f(x) = \left\{
      \begin{array}{cc}
        x^x & (x >0)\\
        -x + A & (x \leqq 0)
      \end{array}
    \right.$

  \item $f(x) = \left\{
      \begin{array}{cc}
        \dfrac{ x-\sin x}{x^3} & (x \neq 0)\\
        A & (x = 0)
      \end{array}
    \right.$

  \item $\ds f(x) = \left\{
      \begin{array}{cc}
        \dfrac{(1-\cos x)\sin x}{x-\sin x} &  (x \neq 0)\\
        A & (x = 0)
      \end{array}
    \right.$
    
  \end{edaenumerate}

\item 次の関数 $f$ が $x=0$ で微分可能となるような定数 $A, B$ の値を求めよう.

  \vspace{1ex}
  
  \begin{edaenumerate}[(1)]
  \item $\ds f(x) = \left\{
      \begin{array}{cc}
        \dfrac{\sin x}{x} & (x > 0)\\
        Ax +B & (x \leqq 0)
      \end{array}
    \right.$

  \item $\ds f(x) = \left\{
      \begin{array}{cc}
        \dfrac{x-\tan^{-1}x}{x^2} & (x >0)\\
        Ax + B & (x \leqq 0)
      \end{array}
    \right.$
    
  \end{edaenumerate}
  
\end{enumerate}


\newpage

\section{ テイラーの定理}

\begin{enumerate}[label=\arabic{section}.\arabic*]

  \setlength{\itemsep}{1zh}
  
\item テイラーの定理を用いて次の値の近似値を誤差精度 $0.01$ 以内で求め,
  小数表示しよう.なお,$\pi = 3.141592653589793\cdots$ であることは適
  宜利用しよう.

  \vspace{1ex}

  \begin{edaenumerate}<retusuu=5>[(1)]

  \item $\sin 3$

  \item $\cos 1.6$

  \item $\sqrt{e}$

  \item $\log 1.2$

  \item $\ds \tan^{-1} 0.02$
    
  \end{edaenumerate}

\item $\ds f(x) = \log\frac{1+x}{1-x}$ をうまく利用して,以下の値の小数
  点以下第3位までを確定させよう.

  \vspace{1ex}

  \begin{edaenumerate}<retusuu=3>[(1)]

  \item $\log 2$

  \item $\log 3$

  \item $\log 5$
  \end{edaenumerate}

\item $f_n(x) = \sqrt[n]{1+x}$ をうまく利用して,以下の値の小数点
  以下第3位までを確定させよう.

  \vspace{1ex}

  \begin{edaenumerate}<retusuu=3>[(1)]

  \item $\sqrt{1.2}$

  \item $\sqrt[3]{1.01}$

  \item $\sqrt[5]{0.8}$
    
  \end{edaenumerate}
  
\end{enumerate}




\end{document}
