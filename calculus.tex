\documentclass[11pt, uplatex, dvipdfmx, twoside]{jsarticle}
\usepackage{graphicx}
\usepackage{amsmath,amsfonts, bm, braket, setspace, emathEy, enumitem}

\newcommand{\ds}{\displaystyle}
\renewcommand{\dlim}{\lim\limits} %emathEyを使わないなら\newcommand


\resettagform


\pagestyle{plain}



\title{\Huge 微分積分 問題集}

\begin{document}


\maketitle
\thispagestyle{empty}

\newpage



\section{数列の極限}\label{sec:sequence}

\begin{enumerate}[label=\arabic{section}.\arabic*]
  
  \setlength{\itemsep}{1zh}
  
\item 第 $n$ 項が次の式で表される数列の極限を求めよう.

  \vspace{1zh}
  
  \begin{edaenumerate}<retusuu=3>[(1)]

  \item $\ds \frac{\sin n}{n}$

  \item $\ds \frac{1+\left(-1\right)^n}{n}$

  \item $\ds 1+ \frac{1}{2} + \frac{1}{3}+\cdots + \frac{1}{n} $ 

  \end{edaenumerate}

\item 数列 $\{a_n\}$ が次の漸化式と初期条件を満たすとする.
  \[
    a_{n+1} = \frac{a_{n}}{2}+\frac{1}{a_{n}} ,\quad a_1=-2
  \]
  \begin{enumerate}[label=(\arabic*)]
    \setlength{\itemsep}{1ex}


  \item $\{a_n\}$ が上に有界であることを確認しよう
    
  \item $\{a_n\}$ が狭義単調増加であることを確認しよう.

  \item $\{a_n\}$ の極限を求めよう.

  \item $a_2^2\; , a_3^2\; , a_4^4$ を具体的に計算し,小数表示してみよう.
  \end{enumerate}

\item 次の漸化式と初期条件を満たす数列 $\{a_n\}$ の極限を求めよう.
  \[
    a_{n+1} = \frac{2a_{n}}{3} + \frac{1}{a_{n}^2}, \quad a_1 = 3
  \]
\end{enumerate}



\section{関数の極限}\label{sec:limit}


\begin{enumerate}[label=\arabic{section}.\arabic*]

  \setlength{\itemsep}{1zh}
  
\item 次の極限を求めよう.

  \vspace{1ex}

  \begin{edaenumerate}<retusuu=3>[(1)]

  \item $\ds \lim_{x \to +0} \sqrt{x} \sin \frac{1}{x}$

  \item $\ds \lim_{x \to \infty} \frac{\sin x}{x}$

  \item $\ds \lim_{x \to \infty} \frac{x-\tan^{-1} x}{x}$
    
  \end{edaenumerate}

\item 次の関数 $f$ が $x=0$ で連続となるような定数 $A$ の値を求めよう.

  \vspace{1ex}

  \begin{edaenumerate}[(1)]
    
  \item $\ds f(x)=\left\{
      \begin{array}{cc}
        2x+3& (x > 0)\\
        -3x+A & (x \leqq 0)
      \end{array}
    \right.$

  \item $\ds f(x)=\left\{
      \begin{array}{cc}
        e^{-x} & (x >0)\\
        \cos^{-1}x +A & (x \leqq 0)
      \end{array}
    \right.$

  \item $\ds f(x) = \left\{
      \begin{array}{cc}
        \tan^{-1}\left(x-\frac{1}{2}\right) & (x > 0)\\
        \sin^{-1} A & (x \leqq 0)
      \end{array}
    \right.$
    
  \item $\ds f(x) = \left\{
      \begin{array}{cc}
        \dfrac{\sin 3x}{2x} & (x > 0)\\
        x^2+A & (x \leqq 0)
      \end{array}
    \right.$

  \item $\ds f(x) = \left\{
      \begin{array}{cc}
        e^{-\frac{1}{x}} & (x >0)\\
        A & (x \leqq 0)
      \end{array}
    \right.$
    
  \item $\ds f(x) = \left\{
      \begin{array}{cc}
        e^{-\frac{1}{x^2}}& (x \neq 0)\\
        A & (x=0)
      \end{array}
    \right.$
  \end{edaenumerate}
  
  
\end{enumerate}

\newpage

\section{導関数}\label{sec:derivative}

\begin{enumerate}[label=\arabic{section}.\arabic*]

  \setlength{\itemsep}{1zh}
  
\item 次の関数 $f$ が $x=0$ で微分可能か否かを判定し,微分可能な
  ら $f'(0)$ の値を求めよう.

  \vspace{1ex}
  
  \begin{edaenumerate}[(1)]
  \item $\ds f(x) = \left\{
      \begin{array}{cc}
        x \log |x| & (x \neq 0)\\
        0 & (x =0)
      \end{array}
    \right.$

  \item $\ds f(x) = \left\{
      \begin{array}{cc}
        \dfrac{1-\cos x}{x} & (x \neq 0)\\
        0 & (x=0)
      \end{array}
    \right.$

  \item $\ds f(x) = \left\{
      \begin{array}{cc}
        x^2 \sin \dfrac{1}{x} & (x \neq 0)\\
        0 & (x=0)
      \end{array}
    \right.$

  \item $\ds f(x) = \left\{
      \begin{array}{cc}
        \sqrt{|x|} \sin \dfrac{1}{x} & (x \neq 0)\\
        0 & (x =0 )
      \end{array}
    \right.$
  \end{edaenumerate}

\item 次の関数 $f$ が開区間 $(-1,1)$ で $C^2$ 級となるような定数 $A, B, C$ の値を求めよう.
  \[
    f(x) = \left\{
      \begin{array}{cc}
        \cos x & (x > 0) \\
        Ax^2+Bx+C & (x \leqq 0)
      \end{array}
    \right.
  \]

\item 次の関数 $f$ に対し,$3$ 次関数 $g(x) =a_3 x^3+a_2x^2 + a_1 x + a_0$ が
  \[
    f(0) = g(0), \quad f'(0) = g'(0), \quad f''(0) = g''(0), \quad f^{(3)}(0) = g^{(3)}(0)
  \]
  を満たすとき,$g(x)$ の係数 $a_0, a_1, a_2, a_3$ の値を求めよう.

  \vspace{1ex}
  
  \begin{edaenumerate}<retusuu=3>[(1)]
  \item $\ds f(x) = \sin x$
        
  \item $\ds f(x) = e^x$

  \item $\ds f(x) = \log (1+x)$
  \end{edaenumerate}
\end{enumerate}



\section{不定形の極限}\label{sec:hospital}

\begin{enumerate}[label=\arabic{section}.\arabic*]

  \setlength{\itemsep}{1zh}
  
\item 次の関数 $f$ が $x=0$ で連続となるような定数 $A$ の値を求めよう.

  \vspace{1ex}

  \begin{edaenumerate}[(1)]
    
  \item $\ds f(x) = \left\{
      \begin{array}{cc}
        x \log x & (x >0) \\
        A & (x \leqq 0)
      \end{array}
    \right.$
   
  \item $\ds f(x) = \left\{
      \begin{array}{cc}
        x^x & (x >0)\\
        -x + A & (x \leqq 0)
      \end{array}
    \right.$

  \item $f(x) = \left\{
      \begin{array}{cc}
        \dfrac{ x-\sin x}{x^3} & (x \neq 0)\\
        A & (x = 0)
      \end{array}
    \right.$

  \item $\ds f(x) = \left\{
      \begin{array}{cc}
        \dfrac{(1-\cos x)\sin x}{x-\sin x} &  (x \neq 0)\\
        A & (x = 0)
      \end{array}
    \right.$
    
  \end{edaenumerate}

\item 次の関数 $f$ が $x=0$ で微分可能となるような定数 $A, B$ の値を求めよう.

  \vspace{1ex}
  
  \begin{edaenumerate}[(1)]
  \item $\ds f(x) = \left\{
      \begin{array}{cc}
        \dfrac{\sin x}{x} & (x > 0)\\
        Ax +B & (x \leqq 0)
      \end{array}
    \right.$

  \item $\ds f(x) = \left\{
      \begin{array}{cc}
        \dfrac{x-\tan^{-1}x}{x^2} & (x >0)\\
        Ax + B & (x \leqq 0)
      \end{array}
    \right.$
    
  \end{edaenumerate}
  
\end{enumerate}


\newpage

\section{テイラーの定理}\label{sec:taylor}

\begin{enumerate}[label=\arabic{section}.\arabic*]

  \setlength{\itemsep}{1zh}
  
\item テイラーの定理を用いて次の値の近似値を誤差精度 $0.01$ 以内で求め,
  小数表示しよう.なお,$\pi = 3.141592653589793\cdots$ であることは適
  宜利用しよう.

  \vspace{1ex}

  \begin{edaenumerate}<retusuu=5>[(1)]

  \item $\sin 3$

  \item $\cos 1.6$

  \item $\sqrt{e}$

  \item $\log 1.2$

  \item $\ds \tan^{-1} 0.02$
    
  \end{edaenumerate}

\item $\ds f(x) = \log\frac{1+x}{1-x}$ をうまく利用して,以下の値の小数
  点以下第3位までを確定させよう.

  \vspace{1ex}

  \begin{edaenumerate}<retusuu=3>[(1)]

  \item $\log 2$

  \item $\log 3$

  \item $\log 5$
  \end{edaenumerate}

\item $f_n(x) = \sqrt[n]{1+x}$ をうまく利用して,以下の値の小数点
  以下第3位までを確定させよう.

  \vspace{1ex}

  \begin{edaenumerate}<retusuu=3>[(1)]

  \item $\sqrt{1.2}$

  \item $\sqrt[3]{1.01}$

  \item $\sqrt[5]{0.8}$
    
  \end{edaenumerate}

\item 以下の条件を満たす関数 $f$ に対し,$f(1)$ の値の小数点以下第 $4$
  位までを確定させよう.
  \[
    f'(x) = e^{-x^2}, \quad f(0)=0
  \]

\item 以下の条件を満たす関数 $f$ に対し,$f$ の $5$ 次マクローリン多項式
  \[
    P(x) = f(0) + f'(0)x + \frac{f''(0)}{2!}x^2 +
    \frac{f^{(3)}(0)}{3!}x^3 + \frac{f^{(4)}(0)}{4!}x^4 + \frac{f^{(5)}(0)}{5!}x^5
  \]
  を求めよう.また,指定された $a$ に対して $P(a)$ の値を求めよう.
  
  \vspace{1ex}

  \begin{enumerate}[label=(\arabic*)]

    \setlength{\itemsep}{1ex}
    
  \item $f'(x) = f(x), \quad f(0) =1$ \quad $\ds (a=1/2)$

  \item $\ds 2f''(x) +5'f(x) -3 f(x)=0, \quad f(0)=1, \; f'(0)= 1/2$ \quad $(a=1)$

  \item $\ds f(x) + \log\left( f'(x)\right) =0, \; f(0)=0$ \quad $(a=0.2)$
    
  \end{enumerate}

\item 関数 $f$ が以下の条件を満たすとする.
  \[
    f''(x) = -f(x), \quad f(\pi)=0, \; f'(\pi) = -1
  \]
  このとき,$f$ の $x=\pi$ の周りの $4$ 次テイラー多項式
  \[
    P(x) = f(\pi) + f'(\pi) (x-\pi) + \frac{f''(\pi)}{2!}(x-\pi)^2 + \frac{f^{(3)}(\pi)}{3!}(x-\pi)^3
    + \frac{f^{(4)}(\pi)}{4!}(x-\pi)^4
  \]
  と $P(3)$ の値を求めよう.
    
\end{enumerate}


\newpage


\section*{解答}

\begin{enumerate}[label=\ref{sec:sequence}.\arabic*]
  \setlength{\itemsep}{1ex}
  
\item
  \begin{enumerate}[label=(\arabic*)]
    \setlength{\itemsep}{1ex}
    
  \item
    $0 \leqq \left| \frac{\sin n}{n}\right| \leqq \frac{1}{n} \to
    0$ より,はさみうちの原理から $\dlim_{n \to \infty} \frac{\sin
      n}{n} = 0.$

  \item
    $0 \leqq \left| \frac{1+(-1)^n}{n}\right| \leqq \frac{2}{n}
    \to 0$ より,はさみうちの原理から
    $\dlim_{n \to \infty}\frac{1+(-1)^n}{n} =0.$

  \item $k$ を $2^k < n \leqq 2^{k+1}$ となる自然数とする.$n \to
    \infty$ のとき $k \to \infty$
    なので,$1+\frac{1}{2}+\frac{1}{3}+ \cdots + \frac{1}{n} \geqq 1 +
    \frac{1}{2} + \left( \frac{1}{3} + \frac{1}{4}\right) + \cdots +
    \left( \cdots + \frac{1}{2^k}\right) \geqq 1 + \frac{1}{2} +
    \left(\frac{1}{4} + \frac{1}{4}\right) + \cdots + \left(
      \frac{1}{2^k} + \cdots + \frac{1}{2^k}\right) = 1+ \frac{k}{2}
    \to \infty \; (k \to
    \infty)$ より,${\ds \lim_{n \to \infty}} \left(1+\frac{1}{2} +
      \frac{1}{3} + \cdots + \frac{1}{n}\right) = \infty.$
  \end{enumerate}

\item
  \begin{enumerate}[label=(\arabic*)]
    \setlength{\itemsep}{1ex}

  \item $a_1=-2<0$ と漸化式から任意の $n$ で $a_n <0$ なので,$\{a_n\}$ は上に有界で
    ある.
    
  \item まず,$a_2=-\frac{3}{2} >a_1$ である.$n \geqq 2$ に対して
    は $a_{n+1}-a_{n} = \frac{2-a_{n}^2}{2a_n} = -
    \frac{(a_{n-1}^2-2)^2}{4a_{n-1}^2 \cdot 2a_n} \geqq 0$ である.ここ
    で,$a_1$ が有理数なのと漸化式の形から $\{a_n\}$ の各項は有理数であ
    る.従って,$a_{n-1}^2 \neq 2$ なので $a_{n+1}-a_n >0$ である.よっ
    て,$\{a_n\}$ は狭義単調増加である.
    
  \item (1), (2) から $\{a_n\}$ は収束するのでその極限値を $\alpha$ と
    する.$\alpha = \dlim_{n \to \infty} a_{n+1} = \dlim_{n \to
      \infty} \left( \frac{a_n}{2}+\frac{1}{a_n}\right) =
    \frac{\alpha}{2}+\frac{1}{\alpha}$ より,$\alpha^2 = 2$ であ
    る.$a_n <0$ より $\dlim_{n \to \infty} a_n = -\sqrt{2}$.

  \item $a_2^2 = \left(-\frac{3}{2}\right)^2 =
    2.25, \; a_3^2 = \left(-\frac{17}{12}\right)^2 =2.0069\cdots, \;
    a_4^2=\left(-\frac{577}{408}\right)^2 =2.000006\cdots$ 
  \end{enumerate}
  
\item $a_1=3>0$ と漸化式から任意の $n$ で $a_n>0$ なので,$\{a_n\}$ は
  下に有界である.さらに,$n \geqq 2$
  に対して$a_n - a_{n+1} = \frac{a_n^3-3}{3a_n^2} =
  \frac{(8a_{n-1}^3+3)(a_{n-1}^3-3)^2}{27a_{n-1}^6 \cdot 3a_{n}^2}
  \geqq 0$ なので,$\{a_n\}$ は($n\geqq 2$ で)単調減少である.よっ
  て,$\{a_n\}$ は収束するのでその極限値を $\alpha$ とすれば,前問と
  同様に漸化式から $\alpha = \frac{2}{3\alpha}+\frac{1}{a^2}$ が成り立
  つので,$\alpha^3=3$である.従って,$\dlim_{n \to \infty} a_n
  = \sqrt[3]{3}$.
\end{enumerate}

\begin{enumerate}[label=\ref{sec:limit}.\arabic*]

  \setlength{\itemsep}{1ex}
  
\item
  \begin{enumerate}[label=(\arabic*)]
    \setlength{\itemsep}{1ex}
    
  \item
    $0 \leqq \left| \sqrt{x}\sin \frac{1}{x}\right| \leqq \sqrt{x} \to
    0 \; (x \to +0)$ より,$\dlim_{x \to +0} \sqrt{x} \sin \frac{1}{x} =0$.
    
  \item
    $0 \leqq \left| \frac{\sin x}{x} \right| \leqq \frac{1}{x} \to 0
    \; (x \to \infty)$ より,$\dlim_{x \to \infty} \frac{\sin x}{x}=0$.

  \item $\frac{x-\tan^{-1}{x}}{x} = 1 -\frac{\tan^{-1}x}{x}$
    と $\left|\frac{\tan^{-1}x}{x}\right| <
    \frac{\pi/2}{x} \to 0 \; ( x \to \infty)$
    より,$\dlim_{x \to \infty}\frac{x-\tan^{-1}x}{x}=1$.
  \end{enumerate}

\item 
  \begin{enumerate}[label=(\arabic*)]
    \setlength{\itemsep}{1ex}
    
  \item
    $\dlim_{x \to +0} f(x)= 3, \dlim_{x \to -0} f(x) = A =
    f(0)$ より,$A=3$.

  \item
    $\dlim_{x \to +0} f(x)  =1,
    \dlim_{x \to -0} f(x) = \frac{\pi}{2} +A = f(0)$ よ
    り,$A=1-\frac{\pi}{2}$.

  \item
    $\dlim_{x \to +0} f(x) = \tan^{-1}\left(-\frac{1}{2}\right),
    \dlim_{x \to -0} f(x) = \sin^{-1} A = f(0)$ よ
    り,$\tan^{-1}\left(-\frac{1}{2}\right) = \sin^{-1}A$ であ
    る.$X=\tan^{-1}\left(-\frac{1}{2}\right)$ とおくと
    $\tan X = -\frac{1}{2}$ かつ $-\frac{\pi}{2} < X < 0$ なの
    で,$A=\sin \left(\tan^{-1}\left(-\frac{1}{2}\right)\right) = \sin
    X = -\frac{1}{\sqrt{5}}$.

  \item
    $\dlim_{x \to +0} f(x) = \dlim_{x \to +0} \frac{\sin
      3x}{2x}=\frac{3}{2}, \; \dlim_{x \to -0} f(x) = A = f(0)$ よ
    り,$A=\frac{3}{2}$.

  \item
    $\dlim_{x \to +0} f(x) = \dlim_{x \to +0} e^{-\frac{1}{x}} = 0,
    \dlim_{x \to -0} f(x) = A = f(0)$ より,$A=0$.

  \item $\dlim_{x \to 0} f(x) = \dlim_{x \to 0} e^{-\frac{1}{x^2}} = 0, f(0)=A$ より,$A=0$.
  \end{enumerate}

\end{enumerate}

\begin{enumerate}[label=\ref{sec:derivative}.\arabic*]
  \setlength{\itemsep}{1ex}
  
\item
  \begin{enumerate}[label=(\arabic*)]
    \setlength{\itemsep}{1ex}
    
  \item
    $\dlim_{x \to 0} \frac{f(x)-f(0)}{x} = \dlim_{x \to 0}\log|x|
    =-\infty$ より,微分不可能.
    
  \item
    $\dlim_{x \to 0} \frac{f(x)-f(0)}{x} = \dlim_{x \to 0}
    \frac{1-\cos x}{x^2}=\frac{1}{2}$ より,微分可能で $f'(0) =
    \frac{1}{2}$.
    
  \item
    $0 \leqq \left| \frac{f(x)-f(0)}{x}\right| = \left| x \sin
      \frac{1}{x}\right| \leqq |x| \to 0 \; (x \to 0)$ より,微分可能で $f'(0)=0$.
    
  \item $x>0$ で 
    $\frac{f(x)-f(0)}{x} = \frac{1}{\sqrt{x}} \sin \frac{1}{x}$ よ
    り,$\dlim_{x \to +0} \frac{f(x)-f(0)}{x}$ が存在しないから微分不可
    能.
  \end{enumerate}

\item $f$ は $x=0$ で連続なので,$C= f(0) = \dlim_{x \to +0} f(x) =
  \cos 0=1$ である.また,$f$ は $x=0$ で微分可能なので
  $f'(0) = \dlim_{x \to +0} \frac{f(x)-f(0)}{x} = \dlim_{x \to -0}
  \frac{f(x)-f(0)}{x}$
  である.$\dlim_{x \to +0} \frac{f(x)-f(0)}{x} = \dlim_{x \to +0}
  \frac{\cos x -1}{x} = 0, \dlim_{x \to -0} \frac{f(x)-f(0)}{x} =
  \dlim_{x \to -0} (Ax+B) = B$ より $f'(0) = B = 0$ である.さら
  に,$f'$ は $x=0$ で微分可能なので
  $f''(0) = \dlim_{x \to +0} \frac{f'(x)-f'(0)}{x} = \dlim_{x \to -0}
  \frac{f'(x)-f'(0)}{x}$
  である.$\dlim_{x \to +0} \frac{f'(x)-f'(0)}{x} = \dlim_{x \to
    +0}\frac{-\sin x}{x}=-1, \dlim_{x \to -0}
  \frac{f'(x)-f'(0)}{x}=\dlim_{x \to -0}2A = 2A$ より
  $f''(0)=-1, A=-\frac{1}{2}$
  である.最後に,$\dlim_{x \to +0} f''(x) = \dlim_{x \to +0} -\cos x
  = -1, \dlim_{x \to -0} f''(x) = \dlim_{x \to -0}
  \left(-\frac{1}{2}x^2+1\right)'' = -1$ より
  $\dlim_{x \to 0} f''(x) = f''(0)$ なので,$f''$ は $x=0$ で連続である.

\item $g(0)=a_0, \, g'(0)=a_1, \, g''(0)=2a_2, \,
  g^{(3)}(0)=6a_3$ より
  $a_0=g(0)=f(0), \, a_1= g'(0) = f'(0), \, a_2 = g''(0)/2 = f''(0)/2,
  \, a_3=g^{(3)}(0)/6=f^{(3)}(0)/6$ である.
  \begin{enumerate}[label=(\arabic*)]
    \setlength{\itemsep}{1ex}
    
  \item
    $a_0 = \sin 0 =0, \, a_1 = \cos 0 = 1, \, a_2 = -\frac{1}{2}\sin 0 =0, \, a_3 =
    -\frac{1}{6}\cos 0 = -\frac{1}{6}$

  \item
    $a_0 = e^0=1, \, a_1=e^0=1, \, a_2= \frac{e^0}{2}=\frac{1}{2}, \,
    a_3=\frac{e^0}{6}=\frac{1}{6}$

  \item
    $a_0=\log (1+0) =0, \, a_1 = (1+0)^{-1} = 1, \, a_2 = \frac{-(1+0)^{-2}}{2}=
    -\frac{1}{2}, \, a_3 = \frac{2(1+0)^{-3}}{6} = \frac{1}{3}$
  \end{enumerate}

\end{enumerate}

\begin{enumerate}[label=\ref{sec:hospital}.\arabic*]
  \setlength{\itemsep}{1ex}
  
\item
  \begin{enumerate}[label=(\arabic*)]
    \setlength{\itemsep}{1ex}
    
  \item
    $A = f(0) = \dlim_{x \to +0} f(x) = \dlim_{x \to +0} x \log x =
    \dlim_{x \to +0} \frac{\log x}{\frac{1}{x}} = \dlim_{x \to +0}
    \frac{\frac{1}{x}}{-\frac{1}{x^2}} = \dlim_{x \to +0} (-x) =0$

  \item
    $A = f(0) = \dlim_{x \to +0} f(x) = \dlim_{x \to +0} x^x =
    \dlim_{x \to +0} e^{x \log x} = e^0 = 1$

  \item
    $A= f(0) = \dlim_{x \to 0}f(x) = \dlim_{x \to 0} \frac{x-\sin
      x}{x^3} = \dlim_{x \to 0}\frac{1-\cos x}{3x^2}=\dlim_{x \to 0}\frac{\sin x}{6x} =\frac{1}{6}$

  \item
    $A=f(0) = \dlim_{x \to 0} f(x) = \dlim_{x \to 0} \frac{(1-\cos
      x)\sin x}{x-\sin x}=\dlim_{x \to 0} \frac{\cos x - \cos^2 x +
      \sin^2x}{1-\cos x} = \dlim_{x \to 0} \frac{-\sin x + 4 \sin x
      \cos x}{\sin x} = \dlim_{x \to 0} \frac{-\cos x + 4\cos^2 x - 4\sin^2 x}{\cos x} = 3$
    
  \end{enumerate}

\item
  \begin{enumerate}[label=(\arabic*)]
    \setlength{\itemsep}{1ex}
    
  \item $f$ は $x=0$ で連続なので
    $B=f(0) = \dlim_{x \to +0} f(x) = \dlim_{x \to +0}\frac{\sin x}{x}
    = 1$ である.$f$ は $x=0$ で微分可能なので
    $\dlim_{x \to +0} \frac{f(x)-f(0)}{x} = \dlim_{x \to +0}
    \frac{\sin x - x}{x^2} = \dlim_{x \to +0} \frac{\cos x-1}{2x} =
    \dlim_{x \to +0} \frac{-\sin x}{2} = 0$ と $\dlim_{x \to -0}
    \frac{f(x)-f(0)}{x} = \dlim_{x \to -0} A = A$ が一致するから $A=0$.
    
  \item $f$ は $x=0$ で連続なので
    $B=f(0) = \dlim_{x \to +0} f(x) = \dlim_{x \to +0}
    \frac{x-\tan^{-1}x}{x^2} = \dlim_{x \to +0}
    \frac{1-(1+x^2)^{-1}}{2x} = \dlim_{x \to +0}
    \frac{-2x(1+x^2)^{-2}}{2}=0$ である.$f$ は $x=0$ で微分可能なの
    で
    $\dlim_{x \to +0}\frac{f(x)-f(0)}{x} = \dlim_{x \to +0}
    \frac{x-\tan^{-1}x}{x^3} = \dlim_{x \to +0}
    \frac{1-(1+x^2)^{-1}}{3x^2}= \dlim_{x \to
      +0}\frac{1}{3(x^2+1)}=\frac{1}{3}$ と
    $\dlim_{x \to -0} \frac{f(x)-f(0)}{x} = \dlim_{x \to -0} A=A$ が一
    致するから $A=\frac{1}{3}$.
    
  \end{enumerate}
\end{enumerate}

\newpage

\begin{enumerate}[label=\ref{sec:taylor}.\arabic*]
  \setlength{\itemsep}{1ex}
  
\item
  \begin{enumerate}[label=(\arabic*)]

    \setlength{\itemsep}{1ex}
    
  \item $f(x) = \sin x$ とすると $\sin 3 = f(3)$ である.$3$ と $\pi$
    が近いので,$x=\pi$ の周りでの $f$ の $3$
    次近似を用いる.$f(x) = {\ds \sum_{n=0}^{3}}
    \frac{f^{(n)}(\pi)}{n!} (x-\pi)^n +R(x)= -(x-\pi) +
    \frac{(x-\pi)^3}{6} + R(x)$ とすると,テイラーの定理から
    $R(3) = \frac{f^{(4)}(c)}{4!}(3-\pi)^4 = \frac{\sin
      c}{24}(3-\pi)^4, \; 3 < c < \pi$ を満たす実数 $c$ が存在す
    る.$|\sin c| <1$
    なので,$|R(3)| < \frac{(\pi-3)^4}{24}<
    \frac{(3.15-3)^4}{24}=0.00625 < 0.01$
    より $-(3-\pi) + \frac{(3-\pi)^3}{6} = 0.141\cdots$ は誤
    差 $0.01$ 以内の $\sin 3$ の近似値である.(実際には $\pi$ の近似値を何桁ま
    で使うかによってさらに多少の誤差が生じるが省略する.)

  \item $f(x) = \cos x$ とすると $\cos 1.6 = f(1.6)$ であ
    る.$1.6$ と $\frac{\pi}{2}=1.570796\cdots $ が近いので,$x=\frac{\pi}{2}$ の周りで
    の $f$ の $3$
    次近似を用いる.$f(x) = {\ds \sum_{n=0}^{3}}
    \frac{f^{(n)}\left(\frac{\pi}{2}\right)}{n!}\left(x-\frac{\pi}{2}\right)^n + R(x) =
    -\left(x-\frac{\pi}{2}\right) +
    \frac{\left(x-\frac{\pi}{2}\right)^3}{6} + R(x)$ とすると,テイラー
    の定理から
    $R(1.6) = \frac{f^{(4)}(c)}{4!}\left(1.6-\frac{\pi}{2}\right)^4 =
    \frac{ \cos c}{24}\left( 1.6-\frac{\pi}{2}\right)^4 , \,
    \frac{\pi}{2} < c < 1.6$ を満たす実数 $c$ が存在する.$|\cos c|
    <1$
    なので,$|R(1.6)| < \frac{\left(1.6 - \frac{\pi}{2}\right)^4}{24}
    < \frac{(1.6-1.5)^4}{24} < 4.17 \times 10^{-6} < 0.01$ よ
    り $-\left(1.6-\frac{\pi}{2}\right) +
    \frac{\left(1.6-\frac{\pi}{2}\right)^3}{6} =-0.029\cdots$ は誤
    差 $0.01$ 以内の $\cos 1.6$ の近似値である.

  \item $f(x)=e^x$ とすると $\sqrt{e} = f\left(\frac{1}{2}\right)$ であ
    る.$\frac{1}{2}=0.5$ は $0$ に近いので,$x=0$ の周りでの $f$ の $3$
    次近似を用いる.$f(x) = {\ds
      \sum_{n=0}^{3}}\frac{f^{(n)}(0)}{n!}x^n + R(x) = 1 + x +
    \frac{x^2}{2} + \frac{x^3}{6} + R(x)$ とすると,テイラーの定理か
    ら
    $R\left(\frac{1}{2}\right) = \frac{f^{(4)}(c)}{4!}\left(
      \frac{1}{2}\right)^4= \frac{e^c}{384}, \, 0 < c < \frac{1}{2}$ を
    満たす実数 $c$ が存在する.$1< e^c < e^{\frac{1}{2}} < e^1 < 3$ な
    ので,$0< R\left(\frac{1}{2}\right) < \frac{3}{384} =
    \frac{1}{128} < 0.01$ より
    $1+\frac{1}{2} + \frac{\left(\frac{1}{2}\right)^2}{2} +
    \frac{\left(\frac{1}{2}\right)^3}{6} = \frac{79}{48}= 1.6458\cdots$
    は誤差 $0.01$ 以内の $\sqrt{e}$ の近似値である.

  \item $f(x) = \log(1+x)$ とすると,$\log 1.2 = f(0.2)$ である.$x=0$
    の周りでの $f$ の $2$
    次近似を用いる.$f(x) = f(0) + f'(0)x + \frac{f''(0)}{2}x^2 + R(x)
    = x -\frac{x^2}{2} + R(x)$ とすると,テイラーの定理から
    $R(0.2) = \frac{f^{(3)}(c)}{3!}(0.2)^3 = \frac{1}{375(1+c)^3}, \,
    0 < c < 0.2$ を満たす実数 $c$ が存在する.$0< \frac{1}{(1+c)^3}
    <1$ なので,$0< R(0.2) < \frac{1}{375} <0.01$ より
    $0.2 - \frac{0.2^2}{2} = 0.18$ は誤差 $0.01$ 以内の $\log 1.2$ の近
    似値である.

  \item $f(x) = \tan^{-1} x$ とすると.$\tan^{-1} 0.02 = f(0.02)$ であ
    る.$x=0$ の周りでの $f$
    の1次近似を用いる.$f(x) = f(0) + f'(0)x + R(x) = x + R(x)$ とする
    と,テイラーの定理から
    $R(0.02) = \frac{f^{(2)}(c)}{2} (0.02)^2=
    -\frac{c}{1250(c^2+1)^2}, \, 0 < c < 0.02$ を満たす実数 $c$ が存在
    する.$\frac{c}{(c^2+1)^2} < \frac{0.02}{(0.02^2+1)^2} <
    0.01998\cdots < 0.02$
    なので,$|R(0.02)| < \frac{0.02}{1250} = \frac{1}{62500} < 0.01$ よ
    り $0.02$ は $\tan^{-1}0.02$ の誤差 $0.01$ 以内の近似値である.
    
  \end{enumerate}

\item
  \begin{enumerate}[label=(\arabic*)]
    
    \setlength{\itemsep}{1ex}
    
  \item
    
  \item
    
  \item
    
  \end{enumerate}


\item
  \begin{enumerate}[label=(\arabic*)]
    
    \setlength{\itemsep}{1ex}
    
  \item
    
  \item
    
  \item
    
  \end{enumerate}

\item

\item
  \begin{enumerate}[label=(\arabic*)]
    
    \setlength{\itemsep}{1ex}
    
  \item
    
  \item
    
  \item
    
  \end{enumerate}

\item
  
\end{enumerate}

\end{document}
